%%%%%%%%%%%%%%%%%%%%%%%%%%%%%%%%%%%%%%%%%%%%%%%%%%%%%%%%%%%%%%%%%%%%%%%%%%%%
% AGUtmpl.tex: this template file is for articles formatted with LaTeX2e,
% Modified March 2013
%
% This template includes commands and instructions
% given in the order necessary to produce a final output that will
% satisfy AGU requirements.
%
% PLEASE DO NOT USE YOUR OWN MACROS
% DO NOT USE \newcommand, \renewcommand, or \def.
%
% FOR FIGURES, DO NOT USE \psfrag or \subfigure.
%
%%%%%%%%%%%%%%%%%%%%%%%%%%%%%%%%%%%%%%%%%%%%%%%%%%%%%%%%%%%%%%%%%%%%%%%%%%%%
%
% All questions should be e-mailed to latex@agu.org.
%
%%%%%%%%%%%%%%%%%%%%%%%%%%%%%%%%%%%%%%%%%%%%%%%%%%%%%%%%%%%%%%%%%%%%%%%%%%%%
%
% Step 1: Set the \documentclass
%
% There are two options for article format: two column (default)
% and draft.
%
% PLEASE USE THE DRAFT OPTION TO SUBMIT YOUR PAPERS.
% The draft option produces double spaced output.
%
% Choose the journal abbreviation for the journal you are
% submitting to:

% jgrga JOURNAL OF GEOPHYSICAL RESEARCH
% gbc   GLOBAL BIOCHEMICAL CYCLES
% grl   GEOPHYSICAL RESEARCH LETTERS
% pal   PALEOCEANOGRAPHY
% ras   RADIO SCIENCE
% rog   REVIEWS OF GEOPHYSICS
% tec   TECTONICS
% wrr   WATER RESOURCES RESEARCH
% gc    GEOCHEMISTRY, GEOPHYSICS, GEOSYSTEMS
% sw    SPACE WEATHER
% ms    JAMES
% ef    EARTH'S FUTURE
%
%
%
% (If you are submitting to a journal other than jgrga,
% substitute the initials of the journal for "jgrga" below.)

\documentclass[draft,sw]{AGUTeX}

\usepackage{amsmath}
\usepackage{txfonts}
\usepackage{url}
\usepackage{lineno}
 \linenumbers*[1]


%  To add line numbers to lines with equations:
%  \begin{linenomath*}
%  \begin{equation}
%  \end{equation}
%  \end{linenomath*}
%%%%%%%%%%%%%%%%%%%%%%%%%%%%%%%%%%%%%%%%%%%%%%%%%%%%%%%%%%%%%%%%%%%%%%%%%
% Figures and Tables
%
%
% DO NOT USE \psfrag or \subfigure commands.
%
%  Figures and tables should be placed AT THE END OF THE ARTICLE,
%  after the references.
%
%  Uncomment the following command to include .eps files
%  (comment out this line for draft format):
%  \usepackage[dvips]{graphicx}
%
%  Uncomment the following command to allow illustrations to print
%   when using Draft:
%  \setkeys{Gin}{draft=false}
%
% Substitute one of the following for [dvips] above
% if you are using a different driver program and want to
% proof your illustrations on your machine:
%
% [xdvi], [dvipdf], [dvipsone], [dviwindo], [emtex], [dviwin],
% [pctexps],  [pctexwin],  [pctexhp],  [pctex32], [truetex], [tcidvi],
% [oztex], [textures]
%
% See how to enter figures and tables at the end of the article, after
% references.
%
%% ------------------------------------------------------------------------ %%
%
%  ENTER PREAMBLE
%
%% ------------------------------------------------------------------------ %%

% Author names in capital letters:
\authorrunninghead{CHANDORKAR ET AL.}

% Shorter version of title entered in capital letters:
\titlerunninghead{GAUSSIAN PROCESS $Dst$ MODELS}

%Corresponding author mailing address and e-mail address:
\authoraddr{Corresponding author: M. H. Chandorkar,
Multiscale Dynamics, Centrum Wiskunde Informatica, Science Park 123, 1098XG Amsterdam, Netherlands.
(mandar.chandorkar@cwi.nl)}

\begin{document}

%% ------------------------------------------------------------------------ %%
%
%  TITLE
%
%% ------------------------------------------------------------------------ %%


\title{Probabilistic Forecasting of the Disturbance Storm Time Index: An Autoregressive Gaussian Process approach}

\authors{M. Chandorkar,\altaffilmark{1}
 E. Camporeale,\altaffilmark{1}
 S. Wing\altaffilmark{2}}

\altaffiltext{1}{Multiscale Dynamics, Centrum Wiskunde Informatica (CWI), Amsterdam,
              1098XG Amsterdam}

\altaffiltext{2}{The Johns Hopkins University Applied Physics Laboratory, 
              Laurel, Maryland, 20723, USA}

% >> Do NOT include any \begin...\end commands within
% >> the body of the abstract.

\begin{abstract}
We present a methodology for generating probabilistic predictions for the \emph{Disturbance Storm Time} ($Dst$) geomagnetic activity index. We focus on the \emph{One Step Ahead} (OSA) prediction task and use the OMNI hourly resolution data to build our models.

Our proposed methodology is based on the technique of \emph{Gaussian Process Regression} (GPR). Within this framework we develop two models; \emph{Gaussian Process Auto-Regressive} (GP-AR) and \emph{Gaussian Process Auto-Regressive with eXogenous inputs} (GP-ARX). 

We also propose a criterion to aid model selection with respect to the order of auto-regressive inputs. Finally we test the performance of the GP-AR and GP-ARX models on a set of 63 geomagnetic storms between 1998 and 2006 and illustrate sample predictions with error bars for some of these events.

\end{abstract}

%% ------------------------------------------------------------------------ %%
%
%  BEGIN ARTICLE
%
%% ------------------------------------------------------------------------ %%

% The body of the article must start with a \begin{article} command
%
% \end{article} must follow the references section, before the figures
%  and tables.

\begin{article}

%% ------------------------------------------------------------------------ %%
%
%  TEXT
%
%% ------------------------------------------------------------------------ %%

\section{Introduction}


The magnetosphere's dynamics and its associated solar wind driver form a complex dynamical system. It is therefore instructive and greatly simplifying to use representative indices to quantify the state of geomagnetic activity.

Geomagnetic indices come in various forms, they may take continuous or discrete values and may be defined with varying time resolutions. Their values are often calculated by averaging or combining a number of readings taken by instruments around the Earth. Each geomagnetic index is a proxy for a particular kind of phenomenon. Some popular indices are the $K_p$, $Dst$ and the $AE$ index.

\begin{enumerate}
    \item $K_p$: The Kp-index is a discrete valued global geomagnetic storm index and is based on 3 hour measurements of the K-indices \citep{Bartels}. The K-index itself is a three hour long quasi-logarithmic local index of the geomagnetic activity, relative to a calm day curve for the given location.
    
    \item $AE$: The Auroral Electrojet Index, $AE$, is designed to provide a global, quantitative measure of auroral zone magnetic activity produced by enhanced Ionospheric currents flowing below and within the auroral oval \citep{AEIndex}. It is a continuous index which is calculated every hour.
    
    \item $Dst$: A continuous hourly index which measures the weakening of the Earths magnetic field due to ring currents and the strength of geomagnetic storms \citep{DesslerAndParker}. 
\end{enumerate}

%Talk about Burton and friends
For the present study, we focus on prediction of the hourly $Dst$ index which is a straightforward indicator of geomagnetic storms. More specifically, we focus on the \emph{one step ahead} (OSA), in this case one hour ahead prediction of $Dst$ because it is the simplest model towards building long term predictions of geomagnetic response of the Earth to changing space weather conditions. 

The $Dst$ OSA prediction problem has been the subject of several modeling efforts in the literature. One of the earliest models has been presented by \citet{JGR:JGR10260} who calculated $Dst(t)$ as the solution of an \emph{Ordinary Differential Equation} (ODE) which expressed the rate of change of $Dst(t)$ as a combination of two terms: decay and injection $\frac{d Dst(t)}{dt} = Q(t) - \frac{Dst(t)}{\tau}$, where Q(t) relates to the particle injection from the plasma sheet into the inner magnetosphere. 

The \citet{JGR:JGR10260} model has proven to be very influential particularly due to its simplicity. Many subsequent works have modified the proposed ODE by proposing alternative expressions for the injection term $Q(t)$ [see \citet{Wang:Dst}, \citet{JGRA:JGRA14856}]. More recently \citet{Ballatore2014} have tried to generate empirical estimates for the injection and decay terms in Burton's equation.

%Talk about NARMAX Dst
Another important empirical model used to predict $Dst$ is the \emph{Nonlinear Auto-Regessive Moving Average with eXogenous inputs} (NARMAX) methodology developed in \citet{doi:10.1080/00207178908559767}, \citet{GRL:GRL13494}, \citet{GRL:GRL20944}, \citet{JGRA:JGRA18657}, \citet{balikhin:narmax}, \citet{JGRA:JGRA20661} and \citet{JGRA:JGRA50192}. The NARMAX methodology builds models by constructing polynomial expansions of inputs and determines the best combinations of monomials to include in the refined model by using a criterion called the \emph{error reduction ratio} (ERR). The parameters of the so called NARMAX OLS-ERR model are calculated by solving the \emph{ordinary least squares} (OLS) problem arising from a quadratic objective objective function. The reader may refer to \citet{billings2013nonlinear} for a detailed exposition of the NARMAX methodology.

%Talk about neural networks
Yet another family of forecasting methods is based on \emph{Artificial Neural Networks} (ANN) that have been a popular choice for building predictive models. Researchers have employed both the standard \emph{feed forward} and the more specialized \emph{recurrent} architectures. \citet{Lund} proposed an \emph{Elman} recurrent network architecture called Lund $Dst$, which used the solar wind velocity, \emph{interplanetary magnetic field} (IMF) and historical $Dst$ data as inputs. \citet{JGRA:JGRA17461} used recurrent neural networks to predict $Kp$. \citet{SWE:SWE286} originally proposed a \emph{feed forward} network for predicting the $K_p$ index which used the \emph{Boyle coupling function} \citep{boyle1997empirical}. The same architecture is adapted for prediction of $Dst$ in \citet{SWE:SWE286}, popularly known as the Rice $Dst$ model. \citet{pallocchia:hal-00318011} proposed a \emph{neural network} model called EDDA to predict $Dst$ using only the IMF data.

%Talk about need for probabilistic forecasts.
Although much research has been done on prediction of the $Dst$ index, must less has been done on probabilistic forecasting of $Dst$. One such work described in \citet{McPherron:2013} involves identification of high speed solar wind streams using the W-S-A model (see \citet{WSAModel}), using predictions of high speed streams to construct ensembles of $Dst$ trajectories which yield the quartiles of $Dst$ time series.

In this work we propose an inductive model for probabilistic forecating of the OSA $Dst$ using only the OMNI data. We use the \emph{Gaussian Process Regression} methodology to construct auto-regressive models for $Dst$ and show how to perform exact inference in this framework. We further outline a methodology to perform model selection with respect to its free parameters and time histories.  


\section{Methodology: Gaussian Process} \label{sec:method}

\emph{Gaussian Processes} first appeared in machine learning research in \citet{Neal:1996:BLN:525544}, as the limiting case of Bayesian inference performed on neural networks with infinitely many neurons in the hidden layers. Although their inception in the machine learning community is recent, their origins can be traced back to the geo-statistics research community where they are known as \emph{Kriging} methods (\citet{krige1951statistical}). In pure mathematics area \emph{Gaussian Processes} have been studied extensively and their existence was first proven by Kolmogorov's extension theorem (\citet{tao2011introduction}). The reader is referred to \citet{Rasmussen:2005:GPM:1162254} for an in depth treatment of Gaussian Processes in machine learning.

Let us assume that we want to model a process in which a scalar quantity $y$ is described by $y = f(\mathbf{x}) + \epsilon$ where   $f(.): \mathbb{R}^d \rightarrow \mathbb{R}$ is an unknown scalar function of a multidimensional input vector $\mathbf{x} \in \mathbb{R}^d$, and $\epsilon \sim \mathcal{N}(0, \sigma^2)$ is Gaussian distributed noise with variance $\sigma^2$.

A set of labeled data points ${(\mathbf{x}_i, y_i); i = 1 \cdots N}$ can be conveniently expressed by a $N \times d$ data matrix $\mathbf{X}$ and a $N \times 1$ response vector $\mathbf{y}$, as shown in equations (\ref{eq:feat}) and (\ref{eq:labels}).

\begin{align}
  \mathbf{X}  = & \left( \begin{array}{c} \mathbf{x}^{T}_1 \\ \mathbf{x}^{T}_2 \\ \vdots \\ \mathbf{x}^{T}_n \end{array} \right)_{n \times d} \label{eq:feat} \\
  \vspace{2\baselineskip}
  \mathbf{y}  = & \left( \begin{array}{c} y_1 \\ y_2 \\ \vdots \\ y_N \end{array} \right) _{n \times 1} \label{eq:labels}
\end{align}

Our task is to infer the values of the unknown function $f(.)$ based on the inputs $\mathbf{X}$ and the noisy observations $\mathbf{y}$. We now assume that the joint distribution of $f(\mathbf{x}_i), i = 1 \cdots N$ is a multivariate Gaussian as shown in equations (\ref{eq:fvalues}), (\ref{eq:normal}) and (\ref{eq:sto}).

\begin{align}
 \mathbf{f} = & \left( \begin{array}{c} f(\mathbf{x}_1) \\ f(\mathbf{x}_2) \\ \vdots \\ f(\mathbf{x}_N) \end{array} \right) \label{eq:fvalues}\\
 \vspace{2\baselineskip}
 \mathbf{f} | \mathbf{x}_1, \cdots, \mathbf{x}_N \sim & \mathcal{N}\left( \mathbf{\mu}, \mathbf{\Lambda} \right)  \label{eq:normal}\\
 \vspace{2\baselineskip}
 p( \mathbf{f} \ | \ \mathbf{x}_1, \cdots, \mathbf{x}_N) = & \frac{1}{(2\pi)^{n/2} det(\mathbf{\Lambda})^{1/2}} exp \left(-\frac{1}{2} (\mathbf{f} - \mathbf{\mu})^T \mathbf{\Lambda}^{-1} (\mathbf{f} - \mathbf{\mu}) \right) \label{eq:sto}
\end{align}

Here $\mathbf{f}$ is a $N\times 1$ vector consisting of the values $f(\mathbf{x}_i), i = 1 \cdots N$. In equation (\ref{eq:normal}), $\mathbf{f}|\mathbf{x}_1, \cdots, \mathbf{x}_N$ denotes the conditional distribution of $\mathbf{f}$ with respect to the input data (i.e., $\mathbf{X}$) and $\mathcal{N}\left( \mathbf{\mu}, \mathbf{\Lambda} \right)$ represents a multivariate Gaussian distribution with mean vector $\mathbf{\mu}$ and covariance matrix $\mathbf{\Lambda}$. The probability density function of this distribution $p( \mathbf{f} \ | \ \mathbf{x}_1, \cdots, \mathbf{x}_N)$ is therefore given by equation (\ref{eq:sto}).

From equation (\ref{eq:sto}), one can observe that in order to uniquely define the distribution of the process, it is required to specify $\mathbf{\mu}$ and $\mathbf{\Lambda}$. For this probability density to be valid, there are further requirements imposed on $\mathbf{\Lambda}$: 

\begin{enumerate}
      \item Symmetry: $\mathbf{\Lambda}_{ij} = \mathbf{\Lambda}_{ji} \ \forall i,j \in {1, \cdots, N} $ 
      \item Positive Semi-definiteness: $\mathbf{z}^T \mathbf{\Lambda} \mathbf{z} \geq 0 \ \forall \mathbf{z} \in \mathbb{R}^N$  
\end{enumerate}

Inspecting the individual elements of $\mathbf{\mu}$ and $\mathbf{\Lambda}$, we realise that they take the following form.

\begin{align}
      \mu_i = & \mathbb{E}[f(\mathbf{x}_i)] := m(\mathbf{x}_i) \\
      \Lambda_{ij} = & \mathbb{E}[(f(\mathbf{x}_i) - \mu_i)(f(\mathbf{x}_j) - \mu_j)] := K(\mathbf{x}_i, \mathbf{x}_j)
\end{align}

Here $\mathbb{E}$ denotes the expectation (average). The elements of $\mathbf{\mu}$ and $\mathbf{\Lambda}$ are expressed as functions $m(\mathbf{x}_i)$ and $K(\mathbf{x}_i, \mathbf{x}_j)$ of the inputs $\mathbf{x}_i,\ \mathbf{x}_j$. Specifying the functions $m(\mathbf{x})$ and $K(\mathbf{x}, \mathbf{x}')$ completely specifies each element of $\mathbf{\mu}$ and $\mathbf{\Lambda}$ and subsequently the finite dimensional distribution of $\mathbf{f} | \mathbf{x}_1, \cdots, \mathbf{x}_N $. In most practical applications of \emph{Gaussian Processes} the mean function is often defined as $m(\mathbf{x}) = 0$, which is not unreasonable if the data is standardized to have zero mean. \emph{Gaussian Processes} are represented in machine learning literature using the following notation:

\begin{equation}
    f(\mathbf{x}) \sim \mathcal{GP}(m(\mathbf{x}), K(\mathbf{x}, \mathbf{x}'))
\end{equation}

\subsection{Inference and Predictions} \label{sec:inference}

Our aim is to infer the function $f(\mathbf{x})$ from the noisy training data and generate predictions $f(\mathbf{x}^{*}_i)$ for a set of test points $ {\mathbf{x}^{*}_i : \forall i \in 1, \cdots, M} $. We define $\mathbf{X}^*$ as the test data matrix whose rows are formed by $\mathbf{x}^{*}_i$ as shown in equation (\ref{eq:testfeat}). 
\begin{equation}
    \mathbf{X}_* = \left( \begin{array}{c} (\mathbf{x}^{*}_1)^T \\ (\mathbf{x}^{*}_2)^T \\ \vdots \\ (\mathbf{x}^{*}_M)^T \end{array} \right)_{M \times d} \label{eq:testfeat} 
\end{equation}

Using the multivariate Gaussian distribution in equation (\ref{eq:sto}) we can construct the joint distribution of $f(\mathbf{x})$ over the training and test points. The vector of training and test outputs $\left( \begin{array}{c} \mathbf{y} \\ \mathbf{f_*} \end{array} \right)$ is of dimension $(N+M) \times 1$ and is constructed by appending the test set predictions $\mathbf{f}_*$ to the observed noisy measurements $\mathbf{y}$.

\begin{align}
    \mathbf{f}_* = & \left( \begin{array}{c} f(\mathbf{x^{*}_1}) \\ f(\mathbf{x^{*}_2}) \\ \vdots \\ f(\mathbf{x^{*}_M}) \end{array} \right)_{M \times 1} \\
     \vspace{4\baselineskip}
    \left( \begin{array}{c} \mathbf{y} \\ \mathbf{f_*} \end{array} \right) | \ \ \mathbf{X}, \mathbf{X}_* \sim & 
    \mathcal{N}\left(\mathbf{0}, \left[ \begin{array}{cc} \mathbf{K} + \sigma^{2} \mathbf{I} & \mathbf{K}_{*} \\ \mathbf{K}_{*}^T & \mathbf{K}_{**} \end{array} \right ] \right) \label{eq:dist}
\end{align}

Since we have noisy measurements of $f$ over the training data, we add the noise variance $\sigma^2$ to the variance of $f$ as shown in (\ref{eq:dist}). The block matrix components of the $(N+M) \times (N+M)$ covariance matrix have the following structure.

\begin{enumerate}
      \item $\mathbf{I}$: The $n \times n$ identity matrix.
      \item $\mathbf{K} = [K(\mathbf{x}_i, \mathbf{x}_j)], \ i,j \in 1,\cdots,n$ : Kernel matrix constructed from all couples obtained from the training data.
      \item $\mathbf{K}_{*} = [K(\mathbf{x}_i, \mathbf{x}^{*}_j)], \ i \in 1,\cdots,n ; j \in 1,\cdots,m$ : Cross kernel matrix constructed from all couples between training and test data points.
      \item $\mathbf{K}_{**} = [K(\mathbf{x}^{*}_i, \mathbf{x}^{*}_j)], \ i,j \in 1,\cdots,m$: Kernel matrix constructed from all couples obtained from the test data.
\end{enumerate}

With the multivariate normal distribution defined in equation (\ref{eq:dist}), probabilistic predictions $f_*$ can be generated by constructing the conditional distribution $\mathbf{f_*}|\mathbf{X},\mathbf{y},\mathbf{X_*}$. Since the original distribution of $\left( \begin{array}{c} \mathbf{y} \\ \mathbf{f_*} \end{array} \right) | \ \ \mathbf{X}, \mathbf{X}_*$ is a multivariate Gaussian, conditioning on a subset of elements $\mathbf{y}$ yields another Gaussian distribution whose mean and covariance can be calculated exactly, as in equation (\ref{eq:posterior}) (see \citet{Rasmussen:2005:GPM:1162254}).
\begin{equation}
    \mathbf{f_*}|\mathbf{X},\mathbf{y},\mathbf{X_*} \sim \mathcal{N}(\mathbf{\bar{f}_*}, \Sigma_*)  \label{eq:posterior},
\end{equation}
where
\begin{align}
    \mathbf{\bar{f}_*} = & \mathbf{K}^T_{*} [\mathbf{K} + \sigma^{2} \mathbf{I}]^{-1} \mathbf{y} \label{eq:posteriormean} \\
    \Sigma_* = & \mathbf{K}_{**} - \mathbf{K}^T_{*} \left(\mathbf{K} + \sigma^{2} \mathbf{I}\right)^{-1} \mathbf{K}_{*} \label{eq:posteriorcov}
\end{align}

The practical implementation of \emph{Gaussian Process} models requires the inversion of the training data kernel matrix $[\mathbf{K} + \sigma^{2} \mathbf{I}]^{-1}$ to calculate the parameters of the predictive distribution $\mathbf{f_*}|\mathbf{X},\mathbf{y},\mathbf{X_*}$. The computational complexity of this inference is dominated by the linear problem in Eq. (\ref{eq:posteriormean}), which can be solved via Cholesky decomposition, with a time complexity of $O(N^3)$, where $N$ is the number of data points.

The distribution of $\mathbf{f_*}| \mathbf{X},\mathbf{y},\mathbf{X_*}$ is known in Bayesian analysis as the \emph{Posterior Predictive Distribution}. This illustrates a key difference between \emph{Gaussian Processes} and other regression models such as \emph{Neural Networks}, \emph{Linear Models} and \emph{Support Vector Machines}: a \emph{Gaussian Process} model does not generate point predictions for new data but outputs a predictive distribution for the quantity sought, thus allowing to construct error bars on the predictions. This property of Bayesian models such as \emph{Gaussian Processes} makes them very appealing for Space Weather forecasting applications. 

The central design issue in applying \emph{Gaussian Process} models is the choice of the function $K(\mathbf{x}, \mathbf{x}')$. The same constraints that apply to $\mathbf{\Lambda}$ also apply to the function $K$. In machine learning, these symmetric positive definite functions of two variables are known as \emph{kernels}. Kernel based methods are applied extensively in data analysis i.e. regression, clustering, classification, density estimation (see \citet{Scholkopf:2001:LKS:559923}, \citet{hofmann2008}).

\subsection{Kernel Functions}

For the success of a \emph{Gaussian Process} model an appropriate choice of kernel function is paramount. The symmetry and positive semi-definiteness of \emph{Gaussian Process} kernels implies that they represent inner-products between some basis function representation of the data. The interested reader is suggested to refer to \citet{Berlinet2004}, \citet{Scholkopf:2001:LKS:559923} and \citet{hofmann2008} for a thorough treatment of kernel functions and the rich theory behind them. Some common kernel functions used in machine learning include the radial basis function (RBF) kernel $K(\mathbf{x}, \mathbf{y}) = \frac{1}{2} exp(-||\mathbf{x} - \mathbf{y}||^2/l^2)$ and the polynomial kernel $K(\mathbf{x}, \mathbf{y}) = (\mathbf{x}^\intercal \mathbf{y} + b)^d$. 

The quantities $l$ in the RBF, and $b$ and $d$ in the polynomial kernel are known as \emph{hyper-parameters}. Hyper-parameters give flexibility to particular kernel structure, for example $d = 1, 2, 3, \cdots$ in the polynomial kernel represents linear, quadratic, cubic and higher order polynomials respectively. The method of assigning values to the \emph{hyper-parameters} is crucial in the model building process. 

In this study, we construct Gaussian Process regression models with linear kernels as shown in equation (\ref{eq:usedKernel}), for \emph{one step ahead} prediction of the $Dst$ index. Our choice of kernel leaves us with only one adjustable hyper-parameter, the model noise $\sigma$. 

\begin{align}
    K_{osa}(\mathbf{x}, \mathbf{y}) & = K_{mlp}(\mathbf{x}, \mathbf{y}) + K_{st}(\mathbf{x}, \mathbf{y}) \label{eq:usedKernel} \\
    K_{mlp}(\mathbf{x}, \mathbf{y}) & = sin^{-1}(\frac{w\mathbf{x}^\intercal \mathbf{y} + b}{\sqrt{w\mathbf{x}^\intercal \mathbf{x} + b + 1} \sqrt{w\mathbf{y}^\intercal \mathbf{y} + b + 1}}) \\
    K_{st}(\mathbf{x}, \mathbf{y}) & = \frac{1}{1 + ||\mathbf{x} - \mathbf{y}||_{2}^d}
\end{align}

We initialize a grid of values for the model noise $\sigma$ and use the error on a predefined validation data set to choose the best performing value of $\sigma$. While constructing the grid of possible values for $\sigma$ it must be ensured that $\sigma > 0$ so that the kernel matrix constructed on the training data is non-singular.

\section{One Step Ahead Prediction} \label{sec:osa}

Below in equations (\ref{eq:Dst}) - (\ref{eq:GPPoly}) we outline a \emph{Gaussian Process} formulation for \emph{OSA} prediction of $Dst$. A vector of features $\mathbf{x}_{t-1}$ is used as input to an unknown function $f(\mathbf{x}_{t-1})$.

The features $\mathbf{x}_{t-1}$ can be any collection of quantities in the hourly resolution OMNI data set. Generally $\mathbf{x}_{t-1}$ are time histories of $Dst$ and other important variables such as plasma pressure $p(t)$, solar wind speed $V(t)$, $z$ component of the interplanetary magnetic field $B_z(t)$.

\begin{align}
    Dst(t) & =  f(\mathbf{x}_{t-1}) + \epsilon \label{eq:Dst} \\
    \epsilon & \sim  \mathcal{N}(0, \sigma^2) \label{eq:GPNoise} \\
    f(x_t) & \sim  \mathcal{GP}(m(\mathbf{x}_t), K(\mathbf{x}_t, \mathbf{x}_s)) \label{eq:DstGP} \\
    K(\mathbf{x}, \mathbf{y}) & =  K_{osa}(\mathbf{x}, \mathbf{y}) \label{eq:GPPoly}
\end{align}

In the following we consider two choices for the input features $\mathbf{x}_{t-1}$ leading to two variants of \emph{Gaussian Process} regression for $Dst$ time series prediction.

\subsection{Gaussian Process Auto-Regressive (GP-AR)} \label{sec:gpar}

The simplest auto-regressive models for \emph{OSA} prediction of $Dst$ are those that use only the history of $Dst$ to construct input features for model training. The input features $\mathbf{x}_{t-1}$ at each time step are the history of $Dst(t)$ until a time lag of $p$ hours.

\begin{align*}
    \mathbf{x}_{t-1} = & \left(Dst(t-1), \cdots , Dst(t-p+1)\right)
\end{align*}

\subsection{Gaussian Process Auto-Regressive with eXogenous inputs (GP-ARX)} \label{sec:gparx}

Auto-regressive models can be augmented by including exogenous quantities in the inputs $\mathbf{x}_{t-1}$ at each time step, in order to improve predictive accuracy. $Dst$ gives a measure of ring currents, which are modulated by plasma sheet particle injections into the inner magnetosphere during sub-storms. Studies have shown that the substorm occurrence rate increases with solar wind velocity (high speed streams) \citep{Kissinger2011,Newell2016}. Prolonged southward interplanetary magnetic field (IMF) $z$-component ($B_z$) is needed for sub-storms to occur \citep{McPherron1986}. An increase in the solar wind electric field, $VB_z$, can increase the dawn-dusk electric field in the magnetotail, which in turn determines the amount of plasma sheet particle that move to the inner magnetosphere \citep{Friedel2001}. Therefore, our exogenous parameters consist of solar wind velocity, IMF $B_z$, and $VB_z$.   

In this model we choose distinct time lags $p$, $p_{v}$ and $p_{b}$ for $Dst$, $V$ and $B_z$ respectively.
    
\begin{align*}
       \mathbf{x}_{t-1} & = (Dst(t-1), \cdots , Dst(t-p+1), \\
        & \ \ \ \ \  V(t-1), \cdots, V(t-p_{v}+1),\\
        & \ \ \ \ \  B_{z}(t-1), \cdots, B_{z}(t-p_{b}+1))
\end{align*}

\section{Model Training and Validation}

Before running performance bench marks for \emph{OSA} $Dst$ prediction on the storm events in \citet{Ji2012}, training and model selection of \emph{GP-AR} and \emph{GP-ARX} models on independent data sets must be performed. For this purpose we choose segments 00:00 1 January 2008 - 10:00 11 January 2008 for training and 00:00 15 November 2014 - 23:00 1 December 2014 for model selection. 

Although the training and model selection data sets both do not have a geomagnetic storm in them, this would not degrade the performance of \emph{GP-AR} and \emph{GP-ARX} because the linear polynomial kernel describes a non stationary and self similar Gaussian Process. This implies that for two events where the time histories of $Dst$, $V$ and $B_z$ are not close to each other but can be expressed as a diagonal rescaling of time histories observed in the training data, the predictive distribution is a linearly rescaled version of the training data $Dst$ distribution. 

The computational complexity of calculation of the predictive distribution is $O(N^3)$, as discussed in section \ref{sec:inference}. This can limit the size of the covariance matrix constructed from the training data. Note that this computation overhead is paid for every unique assignment to the model hyper-parameters. However, our chosen training set has a size of 250 which is still very much below the computational limits of the method and in our case. Solving equation \ref{eq:posteriormean} on a laptop computer takes less than a second for the training set considered in our analysis. 

The size of the training set has been decided by studying the performance of the models for increasing training sets. We have noticed that using training sets with more than 250 values does not increase the overall performance in \emph{OSA} prediction of Dst. 


%%% End of body of article:

%%%%%%%%%%%%%%%%%%%%%%%%%%%%%%%%
%% Optional Appendix goes here
%
% \appendix resets counters and redefines section heads
% but doesn't print anything.
% After typing \appendix
%
%\section{Here Is Appendix Title}
% will show
% Appendix A: Here Is Appendix Title
%
%%%%%%%%%%%%%%%%%%%%%%%%%%%%%%%%%%%%%%%%%%%%%%%%%%%%%%%%%%%%%%%%
%
% Optional Glossary or Notation section, goes here
%

%%%%%%%%%%%%%%
% Notation -- End each entry with a period.
% \begin{notation}
% Term & definition.\\
% Second term & second definition.\\
% \end{notation}
%%%%%%%%%%%%%%%%%%%%%%%%%%%%%%%%%%%%%%%%%%%%%%%%%%%%%%%%%%%%%%%%
%
%  ACKNOWLEDGMENTS

\begin{acknowledgments}
We acknowledge use of NASA/GSFC's Space Physics Data Facility's OMNIWeb (or CDAWeb or ftp) service, and OMNI data. We also acknowledge the authors of \citet{JGRA:JGRA16300} for their scientific inputs towards understanding the \emph{TL} model and assistance in access of its generated predictions. Simon Wing acknowledges supports from CWI and NSF Grant AGS-1058456 and NASA Grants (NNX13AE12G, NNX15AJ01G, NNX16AC39G).
\end{acknowledgments}

%% ------------------------------------------------------------------------ %%
%%  REFERENCE LIST AND TEXT CITATIONS
%
% Either type in your references using
% \begin{thebibliography}{}
% \bibitem{}
% Text
% \end{thebibliography}
%
% Or,
%
% If you use BiBTeX for your references, please use the agufull08.bst file (available at % ftp://ftp.agu.org/journals/latex/journals/Manuscript-Preparation/) to produce your .bbl
% file and copy the contents into your paper here.
%
% Follow these steps:
% 1. Run LaTeX on your LaTeX file.
%
% 2. Make sure the bibliography style appears as \bibliographystyle{agufull08}. Run BiBTeX on your LaTeX
% file.
%
% 3. Open the new .bbl file containing the reference list and
%   copy all the contents into your LaTeX file here.
%
% 4. Comment out the old \bibliographystyle and \bibliography commands.
%
% 5. Run LaTeX on your new file before submitting.
%
% AGU does not want a .bib or a .bbl file. Please copy in the contents of your .bbl file here.
\bibliographystyle{BibTeX/agufull08}
\bibliography{BibTeX/sample-bib-08}
%\begin{thebibliography}{}

%\providecommand{\natexlab}[1]{#1}
%\expandafter\ifx\csname urlstyle\endcsname\relax
%  \providecommand{\doi}[1]{doi:\discretionary{}{}{}#1}\else
%  \providecommand{\doi}{doi:\discretionary{}{}{}\begingroup
%  \urlstyle{rm}\Url}\fi
%
%\bibitem[{\textit{Atkinson and Sloan}(1991)}]{AtkinsonSloan}
%Atkinson, K., and I.~Sloan (1991), The numerical solution of first-kind
%  logarithmic-kernel integral equations on smooth open arcs, \textit{Math.
%  Comp.}, \textit{56}(193), 119--139.
%
%\bibitem[{\textit{Colton and Kress}(1983)}]{ColtonKress1}
%Colton, D., and R.~Kress (1983), \textit{Integral Equation Methods in
%  Scattering Theory}, John Wiley, New York.
%
%\bibitem[{\textit{Hsiao et~al.}(1991)\textit{Hsiao, Stephan, and
%  Wendland}}]{StephanHsiao}
%Hsiao, G.~C., E.~P. Stephan, and W.~L. Wendland (1991), On the {D}irichlet
%  problem in elasticity for a domain exterior to an arc, \textit{J. Comput.
%  Appl. Math.}, \textit{34}(1), 1--19.
%
%\bibitem[{\textit{Lu and Ando}(2012)}]{LuAndo}
%Lu, P., and M.~Ando (2012), Difference of scattering geometrical optics
%  components and line integrals of currents in modified edge representation,
%  \textit{Radio Sci.}, \textit{47},  RS3007, \doi{10.1029/2011RS004899}.

%\end{thebibliography}

%Reference citation examples:

%...as shown by \textit{Kilby} [2008].
%...as shown by {\textit  {Lewin}} [1976], {\textit  {Carson}} [1986], {\textit  {Bartholdy and Billi}} [2002], and {\textit  {Rinaldi}} [2003].
%...has been shown [\textit{Kilby et al.}, 2008].
%...has been shown [{\textit  {Lewin}}, 1976; {\textit  {Carson}}, 1986; {\textit  {Bartholdy and Billi}}, 2002; {\textit  {Rinaldi}}, 2003].
%...has been shown [e.g., {\textit  {Lewin}}, 1976; {\textit  {Carson}}, 1986; {\textit  {Bartholdy and Billi}}, 2002; {\textit  {Rinaldi}}, 2003].

%...as shown by \citet{jskilby}.
%...as shown by \citet{lewin76}, \citet{carson86}, \citet{bartoldy02}, and \citet{rinaldi03}.
%...has been shown \citep{jskilbye}.
%...has been shown \citep{lewin76,carson86,bartoldy02,rinaldi03}.
%...has been shown \citep [e.g.,][]{lewin76,carson86,bartoldy02,rinaldi03}.
%
% Please use ONLY \citet and \citep for reference citations.
% DO NOT use other cite commands (e.g., \cite, \citeyear, \nocite, \citealp, etc.).

%% ------------------------------------------------------------------------ %%
%
%  END ARTICLE
%
%% ------------------------------------------------------------------------ %%
\end{article}
%
%
%% Enter Figures and Tables here:
%
% DO NOT USE \psfrag or \subfigure commands.
%
% Figure captions go below the figure.
% Table titles go above tables; all other caption information
%  should be placed in footnotes below the table.
%
%----------------
% EXAMPLE FIGURE
%
% \begin{figure}
% \noindent\includegraphics[width=20pc]{samplefigure.eps}
% \caption{Caption text here}
% \label{figure_label}
% \end{figure}


%
% ---------------
% EXAMPLE TABLE
%
%\begin{table}
%\caption{Time of the Transition Between Phase 1 and Phase 2\tablenotemark{a}}
%\centering
%\begin{tabular}{l c}
%\hline
% Run  & Time (min)  \\
%\hline
%  $l1$  & 260   \\
%  $l2$  & 300   \\
%  $l3$  & 340   \\
%  $h1$  & 270   \\
%  $h2$  & 250   \\
%  $h3$  & 380   \\
%  $r1$  & 370   \\
%  $r2$  & 390   \\
%\hline
%\end{tabular}
%\tablenotetext{a}{Footnote text here.}
%\end{table}

% See below for how to make sideways figures or tables.

\end{document}

% IF YOU HAVE MULTI-LINE EQUATIONS, PLEASE
% BREAK THE EQUATIONS INTO TWO OR MORE LINES
% OF SINGLE COLUMN WIDTH (20 pc, 8.3 cm)
% using double backslashes (\\).

% To create multiline equations, use the
% \begin{eqnarray} and \end{eqnarray} environment
% as demonstrated below.
\begin{eqnarray}
  x_{1} & = & (x - x_{0}) \cos \Theta \nonumber \\
        && + (y - y_{0}) \sin \Theta  \nonumber \\
  y_{1} & = & -(x - x_{0}) \sin \Theta \nonumber \\
        && + (y - y_{0}) \cos \Theta.
\end{eqnarray}

%If you don't want an equation number, use the star form:
%\begin{eqnarray*}...\end{eqnarray*}

% Break each line at a sign of operation
% (+, -, etc.) if possible, with the sign of operation
% on the new line.

% Indent second and subsequent lines to align with
% the first character following the equal sign on the
% first line.

% Use an \hspace{} command to insert horizontal space
% into your equation if necessary. Place an appropriate
% unit of measure between the curly braces, e.g.
% \hspace{1in}; you may have to experiment to achieve
% the correct amount of space.


%% ------------------------------------------------------------------------ %%
%
%  EQUATION NUMBERING: COUNTER
%
%% ------------------------------------------------------------------------ %%

% You may change equation numbering by resetting
% the equation counter or by explicitly numbering
% an equation.

% To explicitly number an equation, type \eqnum{}
% (with the desired number between the brackets)
% after the \begin{equation} or \begin{eqnarray}
% command.  The \eqnum{} command will affect only
% the equation it appears with; LaTeX will number
% any equations appearing later in the manuscript
% according to the equation counter.
%

% If you have a multiline equation that needs only
% one equation number, use a \nonumber command in
% front of the double backslashes (\\) as shown in
% the multiline equation above.

%% ------------------------------------------------------------------------ %%
%
%  SIDEWAYS FIGURE AND TABLE EXAMPLES
%
%% ------------------------------------------------------------------------ %%
%
% For tables and figures, add \usepackage{rotating} to the paper and add the rotating.sty file to the folder.
% AGU prefers the use of {sidewaystable} over {landscapetable} as it causes fewer problems.
%
% \begin{sidewaysfigure}
% \includegraphics[width=20pc]{samplefigure.eps}
% \caption{caption here}
% \label{label_here}
% \end{sidewaysfigure}
%
%
%
% \begin{sidewaystable}
% \caption{}
% \begin{tabular}
% Table layout here.
% \end{tabular}
% \end{sidewaystable}
%
%

