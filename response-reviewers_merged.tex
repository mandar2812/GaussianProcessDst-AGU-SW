\documentclass{article}

\usepackage[parfill]{parskip}

\begin{document}

\title{Reply to Reviewers}
\maketitle

The authors would like to thank the reviewers for the various comments and suggestions which greatly helped in improving the quality of the manuscript. 
In the proceeding section they will find the revisions/responses for each concern outlined.

\section*{Reviewer 1}

\textit{
\textbf{Comment 1)}: The authors state the proposed methodology allows to construct error bars on the predictions. I think this can not be considered as a significant advantage because the past-cast performance for any forecasting tool can be used to construct statistical error bars for the prediction.}\\
\\
\textbf{Response 1)}: We do not agree with this assertion because the error bars calculated from a past-cast performance are static with respect to the model inputs. That is not the case with error bars generated from the GP-AR and GP-ARX models. From figures 5-7 in the manuscript it can be seen that the error bars on a prediction depend on the inputs provided, this dependence is quantified via the covariance function. To further elaborate the advantage of generating error bars using the Gaussian Process approach we have added the following paragraph (lines 91-97).\\
\\
\fbox{\parbox{\textwidth}{
A simple way to construct error bars on the predictions of forecasting models is by using the so called \textit{past cast} performance i.e. by calculating the standard deviations of the predictions generated by the model on a hold out data set. One limitation of such an approach is that the variance of the model predictions is computed once and for all. It does not adapt according to the inputs provided to the model. This may lead to overestimation or underestimation of the uncertainty around a given prediction, depending on the prevelant geo-magnetic conditions and the data set used to calculate the \textit{past cast} model performance.}
}\\
\\

{\it \textbf{Comment 2)}: Authors should add that ring current is not the only current that contributes to the Dst evolution. 
The authors should mention the contribution of near -Earth cross-tail current (e.g. Ganushkina, et al., Ann. Geophys. 22, 1317-1334, 2004; Ganushkina, et al.. Ann. Geophys. 28, 123- 140, 2010); partial ring current 
(e.g. Liemohn et al. J. Geophys. Res. 106, 10883-10904, 2001), substorm current wedge (e.g. Munsami, J. Geophys. Res. 105(A12), 27833-27840, 2000); magnetopause current. }\\

\textbf{Response 2)}: We agree with the recommendations. We have added those references.\\
\\

\textit{
\textbf{Comment 3)}: NARMAX similar to NARX modelling is not limited to the polynomial based approach. It could employ wide range of sets of functions that form basis e.g. radial basis functions, wavelets etc. (e.g.: Wei et al., Wavelet based non-parametric NARX models for nonlinear input-output system identification; International Journal of Systems Science, Volume: 37, Issue: 15, Pages: 1089-1096, DOI: 10.1080/00207720600903011, Published: DEC 15 2006; Wei et al., Prediction of the Dst index using multiresolution wavelet models; JGR-SP, 109, A7, DOI: 10.1029/2003JA010332, 2004). Please modify the statement about NARMAX. 
}\\

\textbf{Response 3)}: We agree with the recommendations. The following changes have been made (lines 59-62)\\
\\
\fbox{%
\parbox{\textwidth}{It must be noted that the NARMAX methodology is not limited to polynomial functions, rather any set of basis function expansions can be used with it, such as radial basis functions, wavelets etc [Wei et al. 2006; 2004]	.}}\\
\\

\textit{
\textbf{Comment 4)}: V and Bz are used in the current manuscript for GP-ARX model. It is well known that P dynamic pressure affects the Dst. Please discuss if the un-accounted inputs like P can affect statistical properties of "noise" and if these un-accounted inputs can affect the structure of the model.}\\ 

\textbf{Response 4)}: We appreciate this comment as it raises important modelling questions. Our general understanding is that unaccounted solar wind parameters would increase the magnitude of the noise component in the model, on the other hand we do not have strong reasons to think that it will drastically change the structure of the model with respect to parameters such as the autoregressive lag of $Dst$ and the exogenous inputs. We feel this because the solar wind speed and IMF $B_z$ contain a large amount of statistical information with regards to the dynamics of $Dst$. Also worth noting is that the solar wind dynamic pressure in the OMNI data is calculated as a product of plasma density and square of solar wind speed. Revisions have been made along these ideas. \\ 

Lines 229-233:\\
\fbox{%
\parbox{\textwidth}{While modeling $Dst$ using the OMNI data, one must choose which solar wind quantities to include in the exogenous inputs of the predictive model. This choice is not straight forward and eventually requires a compromise between including important solar wind quantities and keeping the input space managable in the interest of simplicity.
}}\\

Lines 256-265:\\
\fbox{%
\parbox{\textwidth}{It is an important question as to how unaccounted inputs such as solar wind dynamic pressure $P$ and clock angle $\theta$ affect the structure of the GP-ARX model. From a model selection perspective, these unaccounted inputs should lead to higher values of the noise covariance. In the specific case of solar wind dynamic pressure, it is calculated as a product of the plasma density and the solar wind speed making it highly correlated with the solar wind speed, as a result the GP-ARX model can infer a large portion of the information content from the solar wind speed itself. With respect to the clock angle, it must be noted that coupling functions such as Akasofu $\epsilon$ generally contain powers of $sin \theta$ bounding the effect of clock angle to an absolute magnitude of $1$, hence we do not expect these unaccounted inputs to greatly improve the predictive capabilities of the GP-ARX model.
 }}\\

\textit{
\textbf{Comment 5)}: There were many previously published papers that tried to identify the coupling functions between the solar wind and the Dst index. Since the introduction of Acasofu "epsilon" parameter many coupling functions include also "clock angle". Please explain in more detail why it was decided not to include "clock angle" into the set of parameters. 
}\\
 


\textbf{Response 5)}: With regards to coupling functions, they can be approximated in the GP-ARX model as long as the eigen-space of the chosen covariance allows functional expressions which resemble common solar wind magnetosphere coupling functions. It is also the case that a large portion of the magnitude of these coupling functions consists of the solar wind parameters already included in the GP-ARX model. Since the clock angle enters these coupling functions through the term $sin^4 \theta$, its magnitude and impact is bounded. Therefore we felt it was not imperative to not include it in the model exogenous inputs. Revisions made in lines 243-253:\\

\fbox{%
\parbox{\textwidth}{Apart from $V$ and $B_z$, other quantities which have been shown to correlate with geomagnetic activity are solar wind dynamic pressure $P$, clock angle $tan \theta = \frac{B_y}{B_z}$, Akasofu $\epsilon$ [Pudovkin and Semenov, 1986] and solar wind magnetosphere coupling functions [Spencer et al. 2011].}}\\


\textit{
Line 219: "It is known that the so called persistence model Dˆ st(t) = Dst(t − 1) performs quite well in the context of OSA prediction".\\ 
\textbf{Comment 6)}: I think the major aim of various Dst forecasting tools is to predict in advance sever geomagnetic storms. Persistence model will fail exactly at the moment when accurate prediction is required, one step before the initiation of major geomagnetic storms. If some statistical parameters show that the persistence model performs quite well, thats mean that these parameters are not adequate in the assessment of forecasting tools and some other parameters should be used. Please modify the phrase above. 
}\\


\textbf{Response 6)}: We agree with this comment, the persistence model will fail one moment before a storm. Since it was not the aim of this research to suggest new performance metrics, we chose to use the ones which are widely used in publications so far. Revisions made in lines 272-273:\\
\\
\fbox{%
\parbox{\textwidth}{In the case of the $Dst$ time series, it is known that the so called \emph{persistence model} $\hat{D}st(t) = Dst(t-1)$ has high correlation with \emph{OSA} $Dst$. Due to its simplicity, we choose the \emph{persistence model} as the prior mean function in our OSA Dst models.} 
}\\
Lines 344-351:\\ 
\\
\fbox{%
\parbox{\textwidth}{
Although it is not a robust predictor for the onset of intense geomagnetic storms, the \emph{persistence model} performs well on classical error metrics such as \emph{root mean square error} and such. From the considerations above, it is quite evident that classical performance metrics are not adequate for model evaluation, nevertheless in space weather literature, metrics such as \emph{RMSE} are very commonly used to compare predictive performance of models. Although not the research focus of this study, we note that there exists a need for the formulation of more informative performance metrics for measurement of predictive performance of geomagnetic predictive models.}}

\textit{
\textbf{Comment 7)}: Figures 5,6,7. Please expand the time interval shown. The figure should include the reasonable period of quiet time before and after the geomagnetic storm.
}\\
 
\textbf{Response 7)}: Changes made for the figures 5-7

\newpage
\section*{Reviewer 2}

\textit{
\textbf{Comment 1)}: 
The authors must be care to review completely the researches from classical to intelligent methods for prediction of Dst, Kp. For example some of references are missing is: 
(a)- Javad Sharifie, Caro Lucas, and Babak N. Araabi, "Locally linear neurofuzzy modeling and prediction 
of geomagnetic disturbances based on solar wind conditions", Space Weather, Vol 4, 2006 
(b)- Javad Sharifi, Babak N. Araabi, and Caro Lucas, "Multi-step prediction of Dst index using singular spectrum analysis and locally linear neurofuzzy modeling", Earth Planets Space, 58, 331-341, 2006 
(c)- Loskutov, A., I. A. Istomin, K. M. Kuzanyan, and O. L. Kotlyarov, Testing and forecasting the time series of the solar activity by singular spectrum analysis, Nonlin. Phenomena in Complex Syst., 4(1), 47-57, 2001a. 
(d)- Loskutov, A., I. Istomin, O. Kotlyarov, and K. Kuzanyan, A study of the regularities in Solar magnetic activity by singular spectrum analysis, Astronomy Letters, 27(11), 745-753, 2001b 
And some of other which especially used for Dst predition from one step to multi-step.
}

\textbf{Response 1)}: We thank the reviewer for pointing out these publications. Revisions made in lines 74-85:\\
\\
\fbox{%
\parbox{\textwidth}{
Apart from the NARMAX and neural network approaches, fuzzy methods have also been applied for $Dst$ prediction, Sharifie et al. [2006] outline the application of \emph{Local Neurofuzzy} models for one hour and two hour predictions of $Dst$ respectively. Local neuro-fuzzy models reduce the input space into a number of regions each with its own expert predictor. The combined model predicts $Dst$ for a new point as a linear combinations of the predictions from each expert weighted by a fuzzy score signifying the importance of each model for the provided input. For improving predictive performance of two hour $Dst$ forecasts in Sharifie et al. [2006], the authors use \emph{singular spectrum analysis} (SSA). Singular spectrum analysis consists of extracting orthogonal components from a lagged time series, it is equivalent to \emph{principal component analysis} (PCA) which is quite extensively used in the machine learning community. Luskutov et al. [2001a, 2001b] provide a good background to the theory and application of SSA to geomagnetic time series.}}\\

\textit{
\textbf{Comment 2)}: To select best number of time lags of input regressor vector, you must use structure identification. To do this, an error criteria must be obtain based on number of time lags of each physical component. The error criteria has a minimum which is appropriate number of lagged time series. }\\

\textbf{Response 2)}: In section 4, we have specified that the model lags for each input component are chosen based on the predictive performance of each generated model on the validation data set. The storm events which comprise the validation data set are specified in table 4 of the manuscript. It is based on this method that we choose the time lags for each input. Care has been taken that no events of the validation data set overlap with the training or test data to ensure unbiased model lag selection.\\

\textit{
\textbf{Comment 3)}: In this paper, even for one step, has one-step time delay, which means that prediction is near Markov (The last observed data is the prediction of next time step), it is obvious from figures 5-7. Then the prediction results is useless. To enhance this, some corrections is in front of authors: they may correct the method to non-Markovian OSA, or use the hybrid prediction methods for example combination of spectrum analysis with OSA or any other hybrid methods.
}\\

\textbf{Response 3)}: The GP-AR and GP-ARX models are strictly speaking non-Markovian models which we have explained in lines 274-281.\\
\fbox{%
\parbox{\textwidth}{The \emph{persistence model} can be described as \emph{Markovian} prediction mechanism, when it is chosen as the prior mean of the GP-AR and GP-ARX model, it is indeed the case that the prior probability distribution of $Dst(t)$ is Gaussian with a strong \emph{Markovian} behavior $P(Dst(t)|\mathbf{x}_t) \sim \mathcal{N}(Dst(t-1), \sqrt{K_{osa}(\mathbf{x}_t, \mathbf{x}_t)})$, but the posterior predictive distribution of $Dst(t)$ conditional on the model training data (given in equation 13) is non-\emph{Markovian} due its dependence on the term denoted by $\mathbf{K}_{*}$ which contains kernel values computed between the test data and training data features. Thus the GP-AR and GP-ARX models when used conditional on training data are non-\emph{Markovian} predictive models.}
}\\

Although many predictive models for $Dst$ are non-Markovian in formulation, due to strong \emph{persistence} behavior of the DST index, the Markovian component is always a dominant portion of all $Dst$ predictions. 

This can be observed in figures 7, 9 in Javad Sharifi, Babak N. Araabi, and Caro Lucas, "Multi-step prediction of Dst index using singular spectrum analysis and locally linear neurofuzzy modeling", Earth Planets Space, 58, 331-341, 2006. 

Moreover, also the NARMAX Dst OSA model as outlined in equation 2 of R. J. Boynton, M. A. Balikhin, S. A. Billings, A. S. Sharma, and O. A. Amariutei, "Data derived NARMAX Dst model", has a strong persistence/Markovian component ($0.8335 Dst(t-1)$) which is more significant as compared to the other terms of the model expansion.

We argue that the persistent behaviour of DST is an instrinsic characteristic of the dataset and not an inductive bias of the $Dst$ models in question.


\textit{
\textbf{Comment 5)}: For enhanced method which author must develop, several extreme geomagnetic storms must be used to validate the method. In this storms the Dst index must be below of -300nT, i.e. Dst $<$ -300nT for example extreme geomagnetic storm of 1989 (Dst goes to about -600nT) and etc. in this way, the author can postulate that have contribution for geomagnetic storm prediction.}\\

\textbf{Response 5)}: With respect to the storm of 1989, there is very little solar wind $V$ and IMF $B_z$ data available for that particular event, hence it was not possible to calculate predictions for it. 
The storms used to evaluate the GP-AR and GP-ARX models are listed on Table 5. Once can notice that some of them are are indeed 'extreme storms' with Dst values below -300. 








\end{document}